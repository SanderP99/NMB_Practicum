\documentclass[a4paper, 12pt, titlepage]{article}

%Taal: Nederlands ("Inhoudsopgave", "Hoofdstuk",...)
\usepackage{graphicx}
\usepackage{subcaption}
\usepackage{amsmath, amssymb, textcomp, mathtools}
\usepackage{geometry}
	\geometry{a4paper, left=20mm, right=20mm, top=20mm, bottom=20mm}

%Hyperlinks
\usepackage{hyperref}

%Opmaak hyperlinks
\hypersetup{colorlinks=false,	urlcolor=cyan,pdfborder=0 0 0}

%geen indents
\setlength\parindent{0pt}

\usepackage{listings}
\lstset{
language=Matlab, % choose the language of the code
%basicstyle=10pt, % the size of the fonts that are used for the code
numbers=left, % where to put the line-numbers
numberstyle=\footnotesize, % the size of the fonts that are used for the line-numbers
stepnumber=1, % the step between two line-numbers. If it's 1 each line will be numbered
numbersep=5pt, % how far the line-numbers are from the code
%backgroundcolor=\color{grey}, % choose the background color. You must add \usepackage{color}
showspaces=false, % show spaces adding particular underscores
showstringspaces=false, % underline spaces within strings
showtabs=false, % show tabs within strings adding particular underscores
frame=single, % adds a frame around the code
%tabsize=2, % sets default tabsize to 2 spaces
%captionpos=b, % sets the caption-position to bottom
breaklines=true, % sets automatic line breaking
breakatwhitespace=false, % sets if automatic breaks should only happen at whitespace
escapeinside={\%*}{*)} % if you want to add a comment within your code
}


\usepackage[dutch]{babel}
\begin{document}

\title{\textbf{Numerieke Modellering en Benadering: Chebyshev veeltermen}}
\author{Sander Prenen}

\date{15 april 2020}
\begin{titlepage}
	\maketitle
	\thispagestyle{empty}
\end{titlepage}

\newpage
\section{Continue kleinste kwadratenbenadering met Chebyshev veeltermen}
In dit practicum wordt geprobeerd continue functies op eindige, re\"ele intervallen te benaderen aan de hand van Chebyshev-veeltermen van de eerste soort. Dit zijn veeltermen die voldoen aan de volgende voorwaarde: $T_k(x) = \cos(k \arccos (x))$ voor $x \in [-1,1]$ en $k = 0,1,2,\ldots$

\subsection{Evalueren van de Chebyshev veeltermen}
De functie \texttt{evalCheb} geeft een vector $v = (f_1, f_2, \ldots, f_N) \in \mathbb{R}^N$ terug. Deze vector wordt bekomen uit inputvectoren $a = (a_0,a_1,\ldots, a_n) \in \mathbb{R}^{n+1}$ en $x = (x_1,x_2,\ldots,x_N) \in \mathbb{R}^N$ op de volgende manier:
$$f_i = y_n(x_i) = a_0T_0(x_i) + a_1T_1(x_i) + \ldots + a_nT_n(x_i)$$

\end{document}